\documentclass[twocolumn,lettersize]{article}

%% \usepackage{fullpage}
\usepackage[margin=1.5cm]{geometry}
\usepackage[dvipsnames]{color}
\definecolor{mygrey}{gray}{0.7}

\title{Spell Descriptions}
\author{Warlock1}
\date{}

\begin{document}

\maketitle


  
    {
  

  \section*{Blade Ward}

   %
    \textit{Abjuration Cantrip} %
   %
  %            
             
  \noindent
  \textbf{Casting Time:} 1 action \\
  \textbf{Duration:} 1 round\\
  \textbf{Range:} Self \\
  \textbf{Components:} V, S 

  You extend your hand and trace a sigil of warding in the air. Until the end of 
    your next turn, you have resistance against bludgeoning, piercing, and slashing 
    damage dealt by weapon attacks.
    

  } %\color

  
    {
  

  \section*{Darkness}

   %
    \textit{Evocation Level 2} %
   %
  %
    (\textit{concentration})%
  %            
             
  \noindent
  \textbf{Casting Time:} 1 action \\
  \textbf{Duration:} Concentration, up to 10 minutes\\
  \textbf{Range:} 60 feet \\
  \textbf{Components:} V, M (Bat fur and a drop of pitch or piece of coal) 

  Magical darkness spreads from a point you choose within range to fill a 15-foot 
    radius sphere for the duration.
    The darkness spreads around corners. A creature 
    with darkvision can't see through this darkness, and nonmagical light can't 
    illuminate it. 
    
    If the point you choose is on an object you are holding or one 
    that isn't being worn or carried, the darkness emanates from the object and 
    moves with it. Completely covering the source of the darkness with an opaque 
    object, such as a bowl or a helm, blocks the darkness.
    
    If any of this spell's 
    area overlaps with an area of light created by a spell of 2nd level or lower, 
    the spell that created the light is dispelled.
    

  } %\color

  
    {
  

  \section*{Fly}

   %
    \textit{Transmutation Level 3} %
   %
  %
    (\textit{concentration})%
  %            
             
  \noindent
  \textbf{Casting Time:} 1 action \\
  \textbf{Duration:} Concentration, up to 10 minutes\\
  \textbf{Range:} Touch \\
  \textbf{Components:} V, S, M (A wing feather from any bird) 

  You touch a willing creature. The target gains a flying speed of 60 feet for the
     duration. When the spell ends, the target falls if it is still aloft, unless it
     can stop the fall.
    
    At Higher Levels: When you cast this spell using a spell 
    slot of 4th level or higher, you can target one additional creature for each 
    slot level above 3rd.
    

  } %\color

  
    {
  

  \section*{Hex}

   %
    \textit{Enchantment Level 1} %
   %
  %
    (\textit{concentration})%
  %            
             
  \noindent
  \textbf{Casting Time:} 1 bonus action \\
  \textbf{Duration:} Concentration, up to 1 hour\\
  \textbf{Range:} 90 feet \\
  \textbf{Components:} V, S, M (The petrified eye of a newt) 

  You place a curse on a creature that you can see within range. Until the spell 
    ends, you deal an extra \texttt{1d6} necrotic damage to the target whenever you hit it 
    with an attack. Also, choose one ability when you cast the spell. The target has
     disadvantage on ability checks made with the chosen ability.
    
    If the target 
    drops to 0 hit points before this spell ends, you can use a bonus action on a 
    subsequent turn of yours to curse a new creature.
    
    A remove curse cast on the 
    target ends this spell early.
    
    At Higher Levels: When you cast this spell using 
    a spell slot of 3rd or 4th level, you can maintain your concentration on the 
    spell for up to 8 hours.
    When you use a spell slot of 5th level or higher, you 
    can maintain your concentration on the spell for up to 24 hours.
    

  } %\color

  
    {
  

  \section*{Levitate}

   %
    \textit{Transmutation Cantrip} %
   %
  %
    (\textit{concentration})%
  %            
             
  \noindent
  \textbf{Casting Time:} 1 action \\
  \textbf{Duration:} Concentration, up to 10 minutes\\
  \textbf{Range:} 60 feet \\
  \textbf{Components:} V, S 

  One creature or object of your choice that you can see within range rises 
    vertically, up to 20 feet, and remains suspended there for the duration. The 
    spell can levitate a target that weighs up to 500 pounds. An unwilling creature 
    that succeeds on a Constitution saving throw is unaffected.
    
    The target can move
     only by pushing or pulling against a fixed object or surface within reach (such
     as a wall or a ceiling), which allows it to move as if it were climbing. You 
    can change the target's altitude by up to 20 feet in either direction on your 
    turn. If you are the target, you can move up or down as part of your move. 
    Otherwise, you can use your action to move the target, which must remain within 
    the spell's range.
    
    When the spell ends, the target floats gently to the ground 
    if it is still aloft.
    

  } %\color

  
    {
  

  \section*{Minor Illusion}

   %
    \textit{Illusion Cantrip} %
   %
  %            
             
  \noindent
  \textbf{Casting Time:} 1 action \\
  \textbf{Duration:} 1 minute\\
  \textbf{Range:} 30 feet \\
  \textbf{Components:} S, M (A bit of fleece) 

  You create a sound or an image of an object within range that lasts for the 
    duration. The illusion also ends if you dismiss it as an action or cast this 
    spell again.
    
    If you create a sound, its volume can range from a whisper to a 
    scream. It can be your voice, someone else's voice, a lion's roar, a beating of 
    drums, or any other sound you choose. The sound continues unabated throughout 
    the duration, or you can make discrete sounds at different times before the 
    spell ends.
    
    If you create an image of an object such as a chair, muddy 
    footprints, or a small chest it must be no larger than a 5-foot cube. The image 
    can't create sound, light, smell, or any other sensory effect. Physical 
    interaction with the image reveals it to be an illusion, because things can pass
     through it.
    
    If a creature uses its action to examine the sound or image, the 
    creature can determine that it is an illusion with a successful Intelligence 
    (Investigation) check against your spell save DC. If a creature discerns the 
    illusion for what it is, the illusion becomes faint to the creature.
    

  } %\color

  
    {
  

  \section*{Scrying}

   %
    \textit{Divination Level 5} %
   %
  %
    (\textit{concentration})%
  %            
             
  \noindent
  \textbf{Casting Time:} 10 minutes \\
  \textbf{Duration:} Concentration, up to 10 minutes\\
  \textbf{Range:} Self \\
  \textbf{Components:} V, S, M (A focus worth at least 1,000 gp, such as a crystal ball, a silver mirror, or a font filled with holy water) 

  You can see and hear a particular creature you choose that is on the same plane 
    of existence as you. The target must make a W isdom saving throw, which is 
    modified by how well you know the target and the sort of physical connection you
     have to it. If a target knows you're casting this spell, it can fail the saving
     throw voluntarily if it wants to be observed.
    
    Knowledge                 Save 
    Modifier
    Secondhand (you have heard of the target) +5
    Firsthand (you have met 
    the target)      +0
    Familiar (you know the target well)     -5
    
    Connection 
                   Save Modifier
    Likeness or picture               -2
    Posession or 
    garment            -4
    Body part, lock of hair, bit of nail, or the like -10
    
    On 
    a successful save, the target isn't affected, and you can't use this spell 
    against it again for 24 hours.
    
    On a failed save, the spell creates an invisible
     sensor within 10 feet of the target. You can see and hear through the sensor as
     if you w ere there. The sensor moves with the target, remaining within 10 feet 
    of it for the duration. A creature that can see invisible objects sees the 
    sensor as a luminous orb about the size of your fist.
    
    Instead of targeting a 
    creature, you can choose a location you have seen before as the target of this 
    spell. When you do, the sensor appears at that location and doesn't move.
    

  } %\color

  
    {
  

  \section*{Speak With Animals}

   %
    \textit{Divination Cantrip} %
   %
  %
    (\textit{ritual})%
  %            
             
  \noindent
  \textbf{Casting Time:} 1 action \\
  \textbf{Duration:} 10 minutes\\
  \textbf{Range:} Self \\
  \textbf{Components:} V, S 

  You gain the ability to comprehend and verbally communicate with beasts for the 
    duration.
    The knowledge and awareness of many beasts is limited by their 
    intelligence, but at minimum, beasts can give you information about nearby 
    locations and monsters, including whatever they can perceive or have perceived 
    within the past day. You might be able to persuade a beast to perform a small 
    favor for you, at the DM's discretion.
    

  } %\color

  
    {
  

  \section*{Speak With Dead}

   %
    \textit{Necromancy Cantrip} %
   %
  %            
             
  \noindent
  \textbf{Casting Time:} 1 action \\
  \textbf{Duration:} 10 minutes\\
  \textbf{Range:} 10 feet \\
  \textbf{Components:} V, S 

  You grant the semblance of life and intelligence to a corpse of your choice 
    within range, allowing it to answer the questions you pose. The corpse must 
    still have a mouth and can't be undead. The spell fails if the corpse was the 
    target of this spell within the last 10 days.
    
    Until the spell ends, you can ask
     the corpse up to five questions. The corpse knows only what it knew in life, 
    including the languages it knew. Answers are usually brief, cryptic, or 
    repetitive, and the corpse is under no compulsion to offer a truthful answer if 
    you are hostile to it or it recognizes you as an enemy. This spell doesn't 
    return the creature's soul to its body, only its animating spirit. Thus, the 
    corpse can't learn new information, doesn't comprehend anything that has 
    happened since it died, and can't speculate about future events.
    

  } %\color

  
    {
  

  \section*{Witch Bolt}

   %
    \textit{Evocation Level 1} %
   %
  %
    (\textit{concentration})%
  %            
             
  \noindent
  \textbf{Casting Time:} 1 action \\
  \textbf{Duration:} Concentration, up to 1 minute\\
  \textbf{Range:} 30 feet \\
  \textbf{Components:} V, S, M (A twig from a tree that has been struck by lightning) 

  A beam of crackling, blue energy lances out toward a creature within range, 
    forming a sustained arc of lightning between you and the target.
    Make a ranged 
    spell attack against that creature. On a hit, the target takes \texttt{1d12} lightning 
    damage, and on each of your turns for the duration, you can use your action to 
    deal \texttt{1d12} lightning damage to the target automatically. The spell ends if you 
    use your action to do anything else. The spell also ends if the target is ever 
    outside the spell's range or if it has total cover from you.
    
    At Higher Levels: 
    When you cast this spell using a spell slot of 2nd level or higher, the initial 
    damage increases by \texttt{1d12} for each slot level above 1st.
    

  } %\color


\end{document}